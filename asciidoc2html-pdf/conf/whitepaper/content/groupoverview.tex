\section{Overview of TSSG Autonomic Management Group}
The \acf{TSSG} is one of the largest integrated \acf{ICT} research centres in Ireland. It is a place of high quality research that addresses the transformation of the telecommunications industry, with a core focus on telecommunications network management, security and mobile services. 

\ac{TSSG} believes that solutions are needed that allow management of the network to evolve so that it is increasingly more automated allowing Network Operators to focus their efforts on the delivery of premium services with assured delivery for the benefit of both Service Providers and Consumers. The ability to increase revenues is fundamental and essential to achieving \acf{RoI}. The \acf{RoI} will ensure continued investment in \acf{CAPEX} in the rollout and upgrade of new network technologies supporting higher bandwidths and greater capabilities – essential for continued innovation, commercialisation and adoption for vertical markets without which would result in stagnation.

Research has been ongoing in seeking to develop new network management solutions that are more efficient, cost effective and scalable. One of the most promising is the vision that networks can organise and manage themselves with reduced intervention from humans. This is commonly known as self-management or self-*. Self-managed networks have the potential benefit of scalability, fast reaction time and self-adaptation to changing network conditions – a fundamental driver for the Future Internet. 

The \ac{NGNM} industry consortium have been defining the requirements for next generation networks for \ac{SON} which are being adopted by the 3GPP as a key driver for 4th Generation mobile networks - \ac{LTE}. 

\ac{TSSG} believes that the introduction of self-* capabilities is a key enabling technology that offers significant potential both in terms of OPEX savings and optimisation of usage-based service operations essential for successful large-scale provisioning and smooth running of premium services. 

\ac{TSSG} performs research to identify the innovative ways of integrating self-managed systems into the network provisioning lifecycle. Core to this research is developing autonomic management solutions incorporating semantic analysis, that can be applied to build federated network and service management systems that understand changes in the environment and coordinate their actions to effectively deliver services on an end-to-end basis.

\ac{TSSG} recognises the importance of reliable self-management necessary to enhance the dependability of the network management systems that are used by network operators. Research and development in key methods and techniques that increase the ability of management systems (specifically self-managed systems) to operate in a way that is aware of the operation of other management systems (legacy or self-managed) and can recover from failure situations exhibited from these systems efficiently and limit disturbances to its network services. Therefore the \ac{TSSG} aims to go beyond the state of the art in this area through research of new architecture capabilities and coordination schemes for fault detection, fault recovery and fault mitigation for self-managed networks.

The \ac{TSSG} has many years experience in advanced Network Management Systems with recent projects including 4WARD and EFIPSANS (FP7 IPs), AutoI (FP7 STREP), MORE (FP6 STREP), MADEIRA and MAGNETO (Celtic Project), AMCNS and FAME (Science Foundation Ireland) and ASYST and ASTRAL (Enterprise Ireland).

\ac{TSSG} is one of the founding members of two of the ETSI \acp{ISG}: (i) \acfp{AFI} and (ii) \acf{MOI}. The \ac{TSSG} also currently holds the positions of Academic Co-chair and Academic Co-chair of the Architecture Expert's Group within the Autonomic Communications Forum (ACF), which is the first international standards body for autonomic communications and holds the secretary position for the IEEE Commmunications Society \ac{CNOM}. 

\begin{table}[h!]
  \begin{center}
    \begin{tabularx}{\textwidth}{ l p{0.4\textwidth} }
		FP7 4WARD & http://www.4ward-project.eu/ \\
		FP7 EFIPSANS & http://www.efipsans.org/ \\
		FP7 AutoI & http://ist-autoi.eu/autoi/ \\
		FP6 MORE & http://www.ist-more.org/ \\
		Celtic Eureka Magneto & http://www.celtic-initiative.org/Projects/MAGNETO/default.asp/ \\
		Celtic Eureka Madeira & http://www.celtic-madeira.org/ \\
		SFI FAME & http://www.fame.ie \\
		ASYST & http://www.asystnm.com/ \\
		ASTRAL & http://www.asystnm.com/astral \\
    \end{tabularx}
  \end{center}
  \caption{Related Projects in the TSSG}
\end{table}

Contact Details:
\begin{verbatim}
	<generic mailing list email account relevant 
	to the technical report subject area>
\end{verbatim}



